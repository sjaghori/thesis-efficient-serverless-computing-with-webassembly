In den letzten Jahren hat sich das serverlose Computing zu einem beliebten Paradigma für die Erstellung skalierbarer und kosteneffizienter Cloud-Anwendungen entwickelt. Dieses Paradigma war jedoch für latenzkritische Anwendungen im IoT-, Mobil- oder Spielesegment aufgrund von Kaltstartlatenzzeiten nicht ausreichend. Zur Lösung dieses Problems hat sich ein Edge-Computing-Paradigma herausgebildet, bei dem die Server der Cloud-Anbieter näher am Kunden platziert werden, um die Latenzzeit zu verbessern. Bei der Verwendung bestehender Virtualisierungstechniken bleibt das Problem der Kaltstarts jedoch eine Herausforderung beim Edge Computing. Darüber hinaus können begrenzte CPU-Leistung und Speicherressourcen in der Host-Umgebung die Latenzzeiten weiter erhöhen, was es schwierig macht, die gewünschte Leistung für latenzkritische Anwendungen zu erreichen. 

Aufgrund dieser Herausforderungen besteht ein Bedarf an einer Ausführungsumgebung, die keine Kaltstartlatenzen aufweist und ebenso sicher ist wie herkömmliche Containerisierungstechnologien und "microVMs" wie Firecracker, die von serverlosen Cloud-Anbietern verwendet werden. Eine vielversprechende Technologie ist WebAssembly. Ursprünglich als gemeinsames Kompilierungsziel im Browser entwickelt, hat sie gezeigt, dass dieselbe Technologie auch in der Serverumgebung eingesetzt werden kann. Sie ist leichtgewichtig, sprachunabhängig, plattformübergreifend, sicher und schnell.

In dieser Arbeit soll untersucht werden, ob die Verwendung der WebAssembly-Technologie den anspruchsvollen Anforderungen des Edge-Computing-Paradigmas gerecht wird und es uns ermöglicht, Anwendungen zu entwickeln, die auf schnelle Antwortzeiten angewiesen sind. Zu diesem Zweck bewerten wir die Leistung von WebAssembly-Engines und berücksichtigen verschiedene Faktoren, die unter anderem die Leistung und die Benutzerfreundlichkeit beeinflussen. 

Wir verwenden sowohl Mikrobenchmarks als auch Makrobenchmarks, um die Leistung von \hbox{WebAssembly}-Laufzeiten und bestehenden serverlosen Angeboten zu bewerten. Durch Mikrobenchmarks finden wir heraus, dass die Instanziierungszeit etwa 40 Mikrosekunden beträgt, verglichen mit Alternativen wie Firecrackers microVM, die etwa 125 Millisekunden benötigt. Die Ergebnisse zeigen, dass die Verwendung von WebAssembly in der Serverless-Umgebung einen bedeutenden Fortschritt bei der Ermöglichung eines breiten Spektrums an latenzkritischen Anwendungen darstellt.
