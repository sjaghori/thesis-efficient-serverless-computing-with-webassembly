In recent years, serverless computing has emerged as a popular paradigm for building scalable and cost-efficient cloud applications. However, this paradigm was not sufficient for latency-critical applications in the IoT, mobile, or gaming segments due to cold-start latencies. An edge computing paradigm has emerged to address this issue, which places cloud provider servers closer to customers to improve latency. However, using existing virtualization techniques, the problem of cold starts remains a challenge in edge computing. Furthermore, limited CPU power and memory resources in the host environment can further increase latencies, making it difficult to achieve the desired performance for latency-critical applications. 

Because of these challenges, there is a need for an execution environment that doesn't have cold start latencies and is as secure as traditional containerization technologies and micro virtual machines such as Firecracker used by serverless cloud providers. One promising technology is WebAssembly. Initially designed as a common compilation target in the browser, it has shown that the same technology can be used in the server environment as well. It is lightweight, language-agnostic, cross-platform, secure, and fast.

The aim of this thesis is to assess whether the usage of WebAssembly technology will satisfy the challenging requirements of the edge computing paradigm, allowing us to build applications that depend on fast response times. For this, we evaluate the performance of WebAssembly engines and consider various factors that affect performance and usability, among other things. 

We use both microbenchmarks and macrobenchmarks to evaluate the performance of \hbox{WebAssembly} runtimes and existing serverless offerings. Through microbenchmarks, we find that the instantiation time is about 40 microseconds compared to alternatives such as Firecracker's microVM, which takes about 125 milliseconds. The results demonstrate that using WebAssembly in the serverless environment represents significant progress towards enabling a wide range of latency-critical applications.