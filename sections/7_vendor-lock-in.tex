Cloud computing's vendor lock-in issue arises when customers become reliant (i.e., locked-in) on a specific cloud provider's technology implementation. This makes it challenging and costly to switch to another vendor due to legal constraints, technical incompatibilities, or significant expenses in the future. The issue of vendor lock-in in cloud computing has been identified as a major challenge because transferring applications and data to other providers is often an expensive and time-consuming process, making portability and interoperability essential considerations \cite{guest_2013_oracle}.

There are two primary factors contributing to the difficulty of achieving interoperability, portability, compliance, trust, and security in cloud computing. The first is the absence of universally adopted standards, APIs, or interfaces that can leverage the ever-changing range of cloud services. The second is the lack of standardized practices for deployment, maintenance, and configuration \cite{oparamartins_2014_critical}, which creates challenges for ensuring consistency and compatibility across cloud environments. 

\section{Key Motivations for shifting to a new Cloud Provider}
There are various reasons why a customer may contemplate changing their service provider:

\begin{enumerate}
    \item Inconsistent or unreliable service quality
    \item Escalating costs associated with function execution, prompting a search for a more cost-effective alternative.
    \item Runtime issues or limitations encountered with the current provider.
    \item The need to integrate a function with a new back-end service that is only offered by a different provider.
    \item Strategic or support considerations within the organization that may necessitate a switch to a different Cloud provider.
\end{enumerate}

\subsection{Service interoperability}


\subsection{Which parts need to be considered when migrating to a different provider?}
To evaluate the feasibility of switching \Gls{FaaS} providers, it is necessary to examine the modifications needed for the function codebase as well as the adjustments required for the execution configuration, deployment, and triggers.

\subsubsection{Differences in handler function}
Each cloud computing service provider has its own unique function signature that must be adhered to for the code to be executed on their respective cloud platforms. These signatures can vary slightly between different providers. The subsequent Javascript code examples illustrate how input can be read from the event and responses can be sent back for each cloud provider. These examples assume that the function is triggered through an HTTP POST request and with a JSON body conaining "myName" property. \\
The first example is an AWS Lambda Function, the body is within the event object, the function resolves the request by returning an object that contains a "statusCode" property along with a body.

\begin{lstlisting}[frame=lines, caption=Basic AWS Lambda Function, captionpos=b, language=JavaScript, showstringspaces=false]
export const handler = async(event, context) => {
    const input = JSON.parse(event.body);
    return {
        statusCode: 200,
        body: JSON.stringify({
            message: `Hello ${input.myName}`,
        })
    };
};
\end{lstlisting}
The next \gls{serverless} function is running on Google Cloud, the difference is not only in the function signature, but also the fact that the function imports the framework.

\begin{lstlisting}[frame=lines, caption=Basic Gen. 2 Google Cloud Functions, captionpos=b, language=JavaScript, showstringspaces=false]
const functions = require("@google-cloud/functions-framework");

functions.http("/", (req, res) => {
  res.status(200).send({
       message: `Hello ${req.body.myName}`
    });
});
\end{lstlisting}

TODO: short description

\begin{lstlisting}[frame=lines, caption=Basic Cloudflare Workers, captionpos=b, language=JavaScript, showstringspaces=false]
export default {
    async fetch(request, env, ctx) {
        const input = JSON.parse(await request.text());
        return new Response(JSON.stringify({
            message: `Hello, ${input.myName}`,
        }), {
            status: 200,
            headers: {
                "content-type": "application/json",
            },
        });
    },
};
\end{lstlisting}

TODO: short description

\begin{lstlisting}[frame=lines, caption=Basic Fastly Compute@Edge, captionpos=b, language=JavaScript, showstringspaces=false]
addEventListener("fetch", (event) => event.respondWith(handleRequest(event)));

async function handleRequest(event) {
  const request = event.request;
  const input = JSON.parse(await request.text());
  return new Response(JSON.stringify({
    message: `Hello, ${input.myName}`,
  }), { 
    status: 200,
    headers: {
      "content-type": "application/json",
    },
  });
}
\end{lstlisting}

\section{Design principles}
\subsection{Facade pattern}
The \gls{facade} pattern can be used as a mitigation strategy to avoid or reduce \gls{serverless} vendor lock-in. By creating a facade layer between the serverless function and the vendor-specific implementation, it is possible to decouple the function from the vendor's specific implementation details.
The facade layer provides a simplified interface that abstracts the underlying vendor implementation, allowing developers to write code that is agnostic to the specific \gls{serverless} vendor. If the vendor needs to be changed, the facade layer can be modified to adapt to the new vendor-specific implementation without affecting the business logic or the interface of the serverless function. This way, the codebase remains modular, and changing vendors becomes a relatively straightforward task.

\subsection{Adapter pattern}
SvelteKit is a modern web framework that effectively demonstrates the use of the Adapter pattern for deployment. Before deploying a SvelteKit application, it must be adapted to the specific deployment target by selecting an appropriate adapter in the configuration. This allows the application to be bundled with the platform-specific configuration \cite{sveltecommunity_2023_adapter}. 
The code snippet below illustrates the adapter configuration of a SvelteKit application, which enables the framework to be adapted to various cloud providers and allows the community to create new adapters. In this case, the deployment target is a Cloudflare Workers \gls{serverless} function.

\begin{lstlisting}[frame=lines, caption=svelte.config.js SvelteKit adapter configuration, captionpos=b, language=JavaScript, showstringspaces=false]
import adapter from '@sveltejs/adapter-cloudflare-workers';
 
/** @type {import('@sveltejs/kit').Config} */
const config = {
  kit: {
    adapter: adapter({
      // adapter options go here
    })
  }
};
 
export default config;
\end{lstlisting}

\section{The role of Wasm Portability}